\section*{Metodología}

La metodología de desarrollo del proyecto será incremental e iterativa. Incremental porque varios componentes y funcionalidades del sistema se desarrollarán en momentos diferentes y serán integradas cuando sean completadas. Iterativa pues se invertirán esfuerzos en revisar constantemente partes del sistema, tanto para mejorar la calidad externa como interna del software\cite{ISOIEC9126}.

Dado que los requerimientos de los interesados pueden cambiar en el transcurso de la ejecución del proyecto, se propone utilizar un enfoque de desarrollo ágil. El mismo contempla la posibilidad de priorización y selección en los alcances y en las prioridades de las diferentes funcionalidades, y fundamentalmente promueve la entrega continua de software y la inclusión de los interesados en el proceso de desarrollo.

\section*{Plan de Tareas}

El proyecto se divide en 5 incrementos. A continuación se da una breve descripción de los mismos:

\paragraph{Incremento 1: Investigación preliminar}\
La primer parte del proyecto consiste en determinar que conjunto de tecnologías serán utilizadas, y elaborará una descripción a alto nivel de las diferentes componentes del sistema. Además se realizará una investigación sobre las técnicas de modelado de comportamiento existentes con el fin de determinar cuales serán implementadas.
\paragraph{Incremento 2: Modelado de comportamiento} \
En esta etapa se implementarán las técnicas de modelado de comportamiento seleccionadas en el incremento anterior. Se utilizarán solo los datos provenientes de la capa de transporte y se desarrollará un software con una interfaz web sencilla donde se muestren los principales parámetros del modelado.
\paragraph{Incremento 3: Detección de anomalías} \
Se implementarán las funcionalidades de detección de anomalías y se agregará al software mecanismos de generación de alarmas. Además se mejorará la interfaz de usuario.
\paragraph{Incremento 4: Modelado de comportamiento de subsistemas} \
Con el fin de mejorar la identificación de anomalías se incorporará al modelo de comportamiento información de los diferentes subsistemas que componen la infraestructura de red.
\paragraph{Incremento 5: Pruebas y documentación}
Se completará el desarrollo del software y se redactará la documentación necesaria. Una vez concluidas las pruebas, se elaborará un informe con los resultados obtenidos

\ \

Al finalizar la primera iteración de cada incremento se obtiene una herramienta de software con las funcionalidades descriptas y calidad de producto final, con excepción de la etapa de investigación preliminar donde se obtendrá un informe en soporte escrito o digital.



\begin{table}[htbp]
	\begin{center}	
		\begin{tabular}{|l|c|}
			\hline 
			Entregable & Fecha de entrega \\ \hline
			Informe de avance 1 & 04/10/2016 \\
			Informe de avance 2 & 08/11/2016 \\
			Informe de avance 3 & 16/12/2016 \\
			Informe de avance 4 & 14/02/2017 \\ \hline
		\end{tabular}
	\end{center}
	\caption{Fechas de entrega de informes de avance.}
	\label{table:informes}
\end{table}

\subsubsection*{Plan de Tareas}

La duración total del proyecto es de $466$ horas, con una dedicación de $20$ horas semanales. A continuación se detalla el plan de tareas.

\begin{enumerate}
	\setlength{\itemsep}{0pt}
	\setlength{\parskip}{0pt}
	\item \textbf{Investigación preliminar} (66hs)
	\begin{enumerate}
		\item Estudio comparativo de las tecnologías y métodos de modelado. (24hs)
		\item Diseño conceptual del sistema. (30hs)
		\item Documentación. (12hs)
	\end{enumerate}
	\item \textbf{Modelado de comportamiento} (84hs)
	\begin{enumerate}
		\item Instalación y configuración de la plataforma de desarrollo. (12hs)
		\item Implementación de funcionalidad de captura de tráfico de red. (24hs)
		\item Implementación de modelado de tráfico. (24hs)
		\item Implementación de interfaz web. (24hs)
	\end{enumerate}
	\item \textbf{Detección de anomalías} (98hs)
	\begin{enumerate}
		\item Implementación de funcionalidad de detección de anomalías. (30hs)
		\item Pruebas de implementación. (24hs)
		\item Implementación de módulo de alarmas. (24hs)
		\item Mejora de interfaz web. (20hs)
	\end{enumerate}
	\newpage
	\item \textbf{Modelado de comportamiento de subsistemas} (146hs)
	\begin{enumerate}
		\item Implementación de funcionalidad de captura de logs de subsistemas. (72hs)
		\begin{itemize}
			\item Servidores web (24hs).
			\item Gestores de base de datos (24hs).
			\item Firewalls (24hs).
		\end{itemize}
		\item Implementación de modelado de comportamiento de subsistemas. (30hs)
		\item Pruebas de implementación. (24hs)
		\item Implementación de funcionalidad de detección de anomalías y alarmas. (20hs)
	\end{enumerate}
	\item \textbf{Pruebas y documentación} (72hs)
	\begin{enumerate}
		\item Pruebas de detección. (20hs)
		\item Elaboración de informe de desempeño. (12hs)
		\item Elaboración de informe final. (40hs)
	\end{enumerate}
\end{enumerate}

\subsection*{Informes de avance}

Se presentarán 4 informes de avance en las fechas de finalización de cada etapa, detalladas en el Cuadro \ref{table:cronograma}. A continuación se detalla que información será incluida en cada informe:

\paragraph{Informe de avance 1} 
Contendrá los resultados obtenidos en los estudios comparativos de las tecnologías y las técnicas de modelado y detección y justificará la elección de las mismas. Además se incluirá una descripción general de la arquitectura del sistema.

\paragraph{Informe de avance 2}
Contendrá información sobre los criterios de selección de características para el modelado de tráfico de red. Se proveerá la guía de instalación y configuración de la plataforma. Además tendrá información sobre cambios realizados en los entregables anteriores y el desempeño del modelo implementado.

\paragraph{Informe de avance 3}
Contendrá información sobre los criterios de selección de características para el modelado del comportamiento de los subsistemas, y el desempeño del modelo implementado. Además se detallarán los cambios realizados en los entregables anteriores.

\paragraph{Informe de avance 4}
Se describirán las pruebas de integración del sistema. Además tendrá información sobre cambios realizados en los entregables anteriores y el desempeño general del sistema. También se incluirán los resultados de las pruebas de detección de la última iteración.

\begin{table}[htbp]
	\begin{center}	
		\begin{tabular}{|l|c|c|c|}
			\hline 
			Etapa & Inicio & Finalización & Duración \\ \hline
			Investigación preliminar & 01/09/2016 & 30/09/2016 & 4 semanas \\
			Modelado de comportamiento & 03/10/2016 & 04/11/2016 & 4 semanas \\
			Detección de anomalías & 07/11/2016 & 16/12/2016 & 5 semanas \\
			Modelado de comportamiento de subsistemas & 19/12/2016 & 18/02/2017 & 8 semanas \\
			Pruebas y documentación & 20/02/2017 & 25/03/2017 & 5 semanas \\ \hline
		\end{tabular}
	\end{center}
	\caption{Fechas estimativas de inicio y fin de actividades.}
	\label{table:cronograma}
\end{table}

\newpage

\subsection*{Diagrama de Gantt}

\begin{figure}[hb]
\begin{center}
\begin{ganttchart}[
	time slot format=isodate,
	x unit=.6mm,
	y unit title=.6cm,
	y unit chart=4mm,
	hgrid,
	link/.style={-latex}
	]{2016-09-01}{2017-03-29}
	\gantttitlecalendar{year, month} \\
	\ganttgroup{Tarea 1}{2016-09-01}{2016-09-30} \\
	\ganttbar[name=a11]{Actividad 1.1}{2016-09-01}{2016-09-08} \\
	\ganttbar[name=a12]{Actividad 1.2}{2016-09-10}{2016-09-30} \\
	\ganttbar[name=a13]{Actividad 1.3}{2016-09-10}{2016-09-30} \\
	\ganttlink{a11}{a12} 
	\ganttlink{a11}{a13} 
	\ganttgroup{Tarea 2}{2016-10-03}{2016-11-04} \\
	\ganttbar[name=a21]{Actividad 2.1}{2016-10-03}{2016-10-06} \\
	\ganttbar[name=a22]{Actividad 2.2}{2016-10-08}{2016-10-16} \\
	\ganttbar[name=a23]{Actividad 2.3}{2016-10-18}{2016-11-04} \\
	\ganttbar[name=a24]{Actividad 2.4}{2016-10-18}{2016-11-04} \\
	\ganttlink{a21}{a22}
	\ganttlink{a22}{a23}
	\ganttlink{a22}{a24} 
	\ganttgroup{Tarea 3}{2016-11-07}{2016-12-16} \\
	\ganttbar[name=a31]{Actividad 3.1}{2016-11-07}{2016-11-15} \\
	\ganttbar[name=a32]{Actividad 3.2}{2016-11-17}{2016-11-26} \\
	\ganttbar[name=a33]{Actividad 3.3}{2016-11-28}{2016-12-16} \\
	\ganttbar[name=a34]{Actividad 3.4}{2016-11-28}{2016-12-16} \\
	\ganttlink{a31}{a32}
	\ganttlink{a32}{a33}
	\ganttlink{a32}{a34}
	\ganttgroup{Tarea 4}{2016-12-19}{2017-02-18} \\
	\ganttbar[name=a41]{Actividad 4.1}{2016-12-19}{2017-01-17} \\
	\ganttbar[name=a42]{Actividad 4.2}{2017-01-19}{2017-02-10} \\
	\ganttbar[name=a43]{Actividad 4.3}{2017-01-19}{2017-02-10} \\
	\ganttbar[name=a44]{Actividad 4.4}{2017-02-12}{2017-02-18} \\
	\ganttlink{a41}{a42}
	\ganttlink{a41}{a43}
	\ganttlink{a42}{a44}
	\ganttlink{a43}{a44}
	\ganttgroup{Tarea 5}{2017-02-20}{2017-03-25} \\
	\ganttbar[name=a51]{Actividad 5.1}{2017-02-20}{2017-02-27} \\
	\ganttbar[name=a52]{Actividad 5.2}{2017-03-06}{2017-03-25} \\
	\ganttbar[name=a53]{Actividad 5.3}{2017-03-06}{2017-03-25}
	\ganttlink{a51}{a52}
	\ganttlink{a51}{a53}
\end{ganttchart}
\end{center}
\caption{Diagrama de Gantt del proyecto}
\end{figure}
%\end{landscape}
