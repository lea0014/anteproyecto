\section*{Metodología}

La metodología de desarrollo del proyecto será incremental e iterativa. Incremental porque varias componentes del sistema se desarrollarán en momentos diferentes y serán integradas cuando sean completadas. Iterativa pues se invertirán esfuerzos en revisar y mejorar partes del sistema.

Con esta metodología se puede dividir el trabajo en incrementos que son revisados constantemente a medida que el proyecto se ejecuta. Luego de una investigación preliminar acerca de las tecnologías disponibles, correspondiente a la primera fase de la ejecución del proyecto, se tiene el punto de partida para comenzar el desarrollo. Dado que es necesario cumplir con los requisitos de \textit{tiempo real}, el incremento que corresponde a las funcionalidades básicas del sistema es revisado constantemente para mejorarlo mientras se desarrollan los incrementos de informes y visualización (ver Plan de Tareas). Una vez todos los incrementos son terminados, son integrados para obtener el producto deseado.

\section*{Plan de Tareas}

El proyecto se divide en 4 incrementos. A continuación se da una breve descripción de los mismos:

\begin{itemize}
\item Incremento 1: Diseño
La primer parte del proyecto consiste en determinar que conjunto de tecnologías serán utilizadas, y elaborar una descripción a alto nivel de las diferentes componentes del sistema y cómo se comunican.
\item Incremento 2: Implementación de captura
En esta etapa se desarrolla el componente que permite capturar la información de entrada y prepararla para procesos posteriores.
\item Incremento 3: Filtrado
Se implementa el módulo que procesa el tráfico de la red y se implementan los filtros.
\item Incremento 4: Visualización
En este incremento se integran las funcionalidades del sistema y se agrega la interfaz de usuario.
\end{itemize}

\subsubsection*{Plan de Tareas}
\begin{enumerate}
	\item Diseño (68hs)
	\begin{enumerate}
		\item Seleccionar tecnologías (16hs)
		\item Diseño conceptual del sistema (40hs)
		\item Documentar diseño (12hs)
	\end{enumerate}
	\item Implementación de captura (72hs)
	\begin{enumerate}
		\item Instalar plataforma de desarrollo (20hs)
		\item Implementar funcionalidad básica de captura (40hs)
		\item Documentar instalación (12hs)
	\end{enumerate}
	\item Filtrado y visualización(200hs)
	\begin{enumerate}
		\item Implementar filtrado en capa de red (60hs)
		\item Implementar filtrado en capa de transporte (60hs)
		\item Diseño de interfaz de usuario (80hs)
	\end{enumerate}
	\item Integración (140hs)
	\begin{enumerate}
		\item Integración de funcionalidades (60hs)
		\item Pruebas de integración (40hs)
		\item Elaboración de informe (40hs)
	\end{enumerate}
\end{enumerate}

