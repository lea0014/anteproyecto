\section*{Riesgos}

En esta sección se enumeran los riesgos identificados, indicadores y estrategia a adoptar según corresponda. Para realizar el análisis cualitativo de los riesgos se asigna una probabilidad de ocurrencia y un impacto a cada riesgo. Luego se priorizan según su \textit{severidad}. Los riesgos identificados son:

\begin{enumerate}
\item \textbf{No se pueden satisfacer las restricciones de performance:} la característica principal del sistema es la posibilidad de procesar tráfico en tiempo real. Puede darse un escenario donde las tecnologías disponibles no permitan alcanzar este requisito.

\textbf{Indicador:} se observan demoras o un progreso lento en las etapas de Núcleo y/o Integración. \\
\textbf{Probabilidad de ocurrencia:} Media. \\
\textbf{Impacto:} Muy alto. \\
\textbf{Estrategia a adoptar:} Mitigar (probabilidad).
Se revisan continuamente los entregables de la etapa de Diseño con el fin de mejorar su calidad.

\item \textbf{No se dispone del hardware necesario:} el servidor de aplicaciones necesario para el desarrollo de la aplicación no se encuentra disponible

\textbf{Indicador:} la etapa de Núcleo no puede comenzar. \\
\textbf{Probabilidad de ocurrencia:} Media. \\
\textbf{Impacto:} Alto. \\
\textbf{Estrategia a adoptar:} Mitigar (impacto).
Se utilizará la infraestructura provista por la facultad.

\item \textbf{Los módulos del sistema no pueden ser integrados correctamente:} aunque cada uno de los módulos cumpla con los requisitos funcionales del sistema, puede ocurrir que el rendimiento se vea degradado producto de los retardos que puede introducir las interfaces de comunicación entre los módulos.

\textbf{Indicador:} se observan demoras o un progreso lento en la etapa de Integración. \\
\textbf{Probabilidad de ocurrencia:} Baja. \\
\textbf{Impacto:} Muy alto. \\
\textbf{Estrategia a adoptar:} Mitigar (probabilidad).
Se revisan continuamente los entregables de la etapa de Integración con el fin de mejorar su calidad.

\item \textbf{Las tecnologías seleccionadas no pueden ser integradas:} se utilizarán un conjunto de tecnologías que deben coexistir para realizar el procesamiento de los datos. Es posible que se encuentren incompatibilidades entre alguna de ellas y no sea posible utilizarlas de forma conjunta.

\textbf{Indicador:} se observan demoras o un progreso lento en la etapa de Diseño. \\
\textbf{Probabilidad de ocurrencia:} Baja. \\
\textbf{Impacto:} Alto. \\
\textbf{Estrategia a adoptar:} Mitigar (impacto). \\
Se realiza un estudio comparativo de varias tecnologías para disponer de alternativas a las determinadas inicialmente.

\item \textbf{El progreso del proyecto no es monitoreado adecuadamente:} es necesario elaborar una cierta cantidad de informes de avance. Puede ocurrir que debido a un monitoreo pobre del avance del proyecto estos documentos estén incompletos para la fecha de entrega.

\textbf{Probabilidad de ocurrencia:} Media. \\
\textbf{Impacto:} Bajo. \\
\textbf{Estrategia a adoptar:} Mitigar (ocurrencia). \\
Los cambios y avances realizados serán registrados mediante un sistema de control de versiones, agregando una breve descripción del trabajo realizado. Estas descripciones serán luego utilizadas para la elaboración de los informes de avance.

\end{enumerate}

\begin{table}
	\centering
	\setlength\tabcolsep{4pt}
	\begin{minipage}{0.4\textwidth}
		\begin{centering}
			\begin{tabular}{|c||c|c|c|c|c|}
				\hline
				& 1 & 2 & 3 & 4 & 5 \\ \hline
				\hline
				1 & \cellcolor[gray]{0.8} & \cellcolor[gray]{0.8} & \cellcolor[gray]{0.8} & \cellcolor[gray]{0.8} & \cellcolor[gray]{0.6} \\ \hline
				2 & \cellcolor[gray]{0.8} & \cellcolor[gray]{0.8} & \cellcolor[gray]{0.6} & \cellcolor[gray]{0.6} & \cellcolor[gray]{0.6} \\ \hline
				3 & \cellcolor[gray]{0.8} & \cellcolor[gray]{0.6} & \cellcolor[gray]{0.6} & \cellcolor[gray]{0.6} & \cellcolor[gray]{0.4} \\ \hline
				4 & \cellcolor[gray]{0.8} & \cellcolor[gray]{0.6} & \cellcolor[gray]{0.6} & \cellcolor[gray]{0.4} & \cellcolor[gray]{0.4} \\ \hline
				5 & \cellcolor[gray]{0.6} & \cellcolor[gray]{0.6} & \cellcolor[gray]{0.4} & \cellcolor[gray]{0.4} & \cellcolor[gray]{0.4} \\ \hline
			\end{tabular}
			\captionof{table}{Probabilidad/impacto.}	
		\end{centering}
	\end{minipage}
	\begin{minipage}{0.4\textwidth}
		\begin{centering}
			\begin{tabular}{|c|l|}
				\hline Severidad & Estrategia \\ \hline
				menor que 4 & Aceptar \\ \hline
				5 a 15 & Mitigar \\ \hline
				16 a 25 & Evitar \\ \hline
			\end{tabular}
			\captionof{table}{Estrategia según severidad.}
		\end{centering}
	\end{minipage}
\end{table}

\newpage
\subsection*{Análisis Riesgos}
En esta sección se muestra el análisis realizado de los riesgos del proyecto, luego se definen las estrategias a adoptar y los riesgos ordenados por importancia según su severidad. \\

\begin{tabular}{|l||c|c|c|}

	\hline
	Riesgo & Imp. & \% ocur. & Sev. \\
	\hline
	No se pueden satisfacer las restricciones de performance.  & 5 & 3 & 15 \\
	No se dispone del hardware necesario. & 4 & 3 & 12 \\
	Los módulos del sistema no pueden ser integrados correctamente. & 5 & 2 & 10 \\
	Las tecnologías seleccionadas no pueden ser integradas. & 4 & 2 & 8 \\
	El progreso del proyecto no es monitoreado adecuadamente. & 2 & 3 & 6 \\
	\hline
	
\end{tabular}
\captionof{table}{Lista de riesgos ordenados por severidad.}


\section*{Presupuesto}

A continuación se detalla el presupuesto necesario para el desarrollo del proyecto. El servidor de desarrollo cuenta con hardware de red específico y será utilizado para ejecutar las pruebas del sistema.


\begin{center}

	\begin{tabular}{l c c r}
		\hline \hline
		
		\textbf{Infraestructura} & & & \\ \hline
		Estación de trabajo (costo de amortización)& & & \$2691. \\
		\textit{Procesador AMD FX6} & & &\\
		\textit{Memoria 16GB DDR3} & & & \\
		\textit{Monitores (configuración dual)} & & & \\
		\hline
		Servidor de desarrollo & & & \$18000. \\
		\textit{Procesador Intel i5 o similar} & & & \\
		\textit{Memoria 16GB DDR3} & & & \\
		\textit{Interfaz de red para servidor (2 puertos)} & & & \\
		\hline \hline
		
		\textbf{Servicios} & & &\\ \hline
		Servicio de Internet & & & \$4000 \\
		\hline \hline
		\textbf{Insumos} & & &\\ \hline
		Resma de Hojas & & & \$80 \\
		Toner & & & \$200 \\
		Artículos de librería & & & \$150 \\
		\hline \hline
		\textbf{Recursos humanos} & \textbf{Costo por hora} & \textbf{Horas} \\
		Diseñador & \$110 & 84 & \$9420 \\
		Desarrollador & \$100 & 364 & \$36400 \\
		Tester & \$85 & 40 & \$3400 \\
		\hline \hline
		\textbf{Costo total} & & & \$\textbf{74341}.
	\end{tabular}
\end{center}



