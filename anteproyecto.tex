\documentclass[a4paper,10pt, oneside]{book}
\usepackage[utf8]{inputenc}
\usepackage[spanish]{babel}
\usepackage[style=ieee,backend=bibtex]{biblatex}
\usepackage{graphicx}
\usepackage{draftwatermark}
\usepackage{lineno}
\usepackage[table]{xcolor}
%\usepackage{showframe}
\usepackage[a4paper]{geometry}
\usepackage{caption}
\usepackage{pgfgantt}
\usepackage{lscape} 


\bibliography{anteproyecto}

\SetWatermarkText{}
\SetWatermarkScale{5}
\SetWatermarkLightness{0.95}

\renewcommand{\labelenumii}{\theenumii}
\renewcommand{\theenumii}{\theenumi.\arabic{enumii}.}


	
\begin{document}
\begin{titlepage}
	\centering
	\includegraphics[width=0.25\textwidth]{Universidad_del_Litoral}\par\vspace{1cm}
	{\scshape\LARGE Universidad Nacional del Litoral \par}
	\vspace{1cm}
	{\scshape\Large Proyecto Final de Carrera\par}
	\vspace{1.5cm}
	{\huge\bfseries Diseño de un sistema de detección de anomalías en redes de computadoras.\par}
	\vspace{2cm}
	{\Large\itshape Pineda Leandro\par}
	\vfill
	dirigido por Ing. Miguel Angel Robledo\par
	codirigido por Ing. Gabriel Filippa
	

	\vfill
	
	%Repositorio git del documento (solo versión borrador): https://github.com/leandropineda/anteproyecto
	
	%\vfill
	
	% Bottom of the page
	\large Santa Fe\par
	{\large \today\par}
	
\end{titlepage}

\modulolinenumbers[5]
%\linenumbers

\section*{Resumen}

En la actualidad las redes informáticas están presentes en nuestro día a día, así como también en las operaciones diarias de prácticamente cualquier organización. Debido a la demanda de contenido multimedia, la telefonía celular y el surgimiento del \textit{BigData}, el tamaño y el volumen de datos que manejan las empresas proveedoras de este tipo de servicios incrementa año tras año. Por esto, las herramientas que facilitan la tarea de administración y diagnóstico de redes son de gran utilidad para ofrecer un servicios de mejor calidad, y aprovechar la utilización de recursos existentes.
El documento describe un proyecto para el desarrollo de un sistema de monitoreo en tiempo real del tráfico de red, que puede ser puesto en funcionamiento en la infraestructura existente de cualquier organización con un bajo impacto y a un bajo costo.

\paragraph{Palabras claves} \textit{tiempo real}, \textit{redes informáticas}, \textit{Data Streaming}, \textit{monitoreo}, \textit{visualización}

\newpage
\section*{Justificación}
La cantidad de servicios que organismos públicos y privados ofrecen a través de Internet aumenta constantemente, y en consecuencia también lo hace la complejidad de la infraestructura y los sistemas necesarios para proveerlos. \par
La integridad de estos sistemas puede verse comprometida principalmente por dos factores: las fallas de los componentes de la infraestructura de red\cite{Gill:2011:UNF:2043164.2018477}, que afectan a una parte o a la totalidad de las funcionalidades del sistema, y los numerosos tipos de ataques\cite{Karumanchi:2014:WLS:2554850.2555010}\cite{mutchler15:mobilewebapps} a los servicios ofrecidos al público. Hacer frente a estos ataques es de extrema importancia dado que su fin puede ser no solo afectar la calidad de los servicios ofrecidos, sino que su objetivo puede ser expropiar información sensible o crítica para desarrollo de las actividades de la organización. Además de ataques externos, los sistemas pueden ser vulnerados por usuarios malintensionados con acceso físico a la red interna de la organización, o por actores inadvertidos\cite{Kraemer2007143}\cite{Kraemer2009509}\cite{Liginlal2009215}\cite{Ahmed12humanerrors}. \par
La utilización de herramientas clásicas para implementar políticas de seguridad como firewall, lista de control de acceso, credenciales de usuario, proxy de red entre otros, resultaba adecuado 10 años atrás, pero en la actualidad estas herramientas están siendo cada vez menos efectivas en la tarea de bloquear ataques dirigidos y malware avanzados. Esto hace que cada vez sea más común la utilización de otros métodos que monitorean continuamente eventos en busca de patrones que puedan indicar comportamientos anómalos en el tráfico de red, provocados tanto por ataques como por fallas en los dispositivos de red. Los sistemas de detección/prevención de intrusiones (IDS o IPS por las siglas en inglés \textit{Intrusion Detection/Prevention System}) pueden realizar esta tarea utilizando dos enfoques: detección de uso indebido o \textit{signature-based detection} y detección de anomalías o \textit{anomaly-based detection}\cite{Milenkoski:2015:ECI:2808687.2808691}. En el primer caso, el sistema determina la ocurrencia de ataques comparando la actividad de la red y de los subsistemas que la componen, contra un conjunto de patrones de ataques conocidos. Si los patrones coinciden, el ataque es informado y pueden tomarse medidas al respecto. En el segundo caso, se modela el comportamiento normal de la red y se usa como base para buscar comportamientos anómalos, los cuales pueden ser provocados por fallas o indicadores de potenciales ataques. \par
Como alternativa a las soluciones de hardware dedicado o sistemas privativos que ofrecen algunas empresas, la comunidad de software libre desarrolla y mantiene la aplicación Snort\footnote{https://www.snort.org/}, un IDS basado en detección de uso indebido. Si bien esta herramienta es efectiva detectando ataques conocidos, es dependiente de la definición de un conjunto efectivo de reglas por parte del administrador (las cuales son difíciles de mantener) y de una base de datos actualizada con los patrones de ataques o \textit{malware}. Otra desventaja de esta técnica de detección es la imposibilidad de detectar ataques \textit{zero-day\footnote{ataques que explotan vulnerabilidades que aún no han sido conocidas o que no están contempladas en al conjunto definido de reglas}}\cite{Milenkoski:2015:ECI:2808687.2808691}. Otra herramienta que suele utilizarse en conjunto con Snort es Bro\footnote{https://www.bro.org/}, una plataforma para análisis de tráfico de red. Sin embargo, Bro también se basa en la definición de reglas de monitoreo y utiliza detección de uso indebido. \par

Con el fin de disponer de otra herramienta para la identificación de comportamientos sospechosos que complemente las funcionalidades de las que se utilizan en la actualidad, el objetivo de este proyecto consiste en diseñar un software que modele de forma automática el comportamiento normal de la red utilizando las técnicas del estado del arte, para identificar y alertar comportamiento anómalo en la infraestructura de red de datos de la Secretaría de Tecnologías para la Gestión de la Provincia de Santa Fe. El modelo de comportamiento normal es calculado automáticamente por el software, brindando la posibilidad de detectar efectivamente cualquier evento poco usual en los sistemas. Para esto, se hará uso de la información generada por los dispositivos de infraestructura de red como routers y los logs de servidores y subsistemas que interactuan dentro de la red.

\section*{Objetivos}
\subsection*{Generales}
Diseñar un sistema de detección de anomalías basado en detección de comportamiento.
\subsection*{Específicos}
\begin{itemize}

\item Identificar un conjunto de tecnologías y herramientas adecuadas para el diseño del sistema.
\item Determinar las técnicas de modelado de tráfico de red necesarias.
\item Implementar un software que modele el tráfico normal de red de forma automática, utilizando técnicas no supervisadas.
\item Implementar un software que genere alertas ante la ocurrencia de eventos anómalos.
\item Diseñar un sistema que minimice la clasificación de eventos como falsos negativos. Este objetivo es de particular importancia dado que un evento anómalo clasificado como falso negativo es un evento potencialmente dañino considerado como normal.
\item Elaborar un documento que describa la arquitectura del sistema.
\end{itemize}

\section*{Alcances}

\subsubsection*{Funcionales}
\begin{itemize}
	\item El sistema utilizará información de capa de red\cite{rfc791} y capa de transporte\cite{rfc793}\cite{rfc1180}, e información proveniente de logs de servidores web y gestores de bases de datos.
	\item Los datos serán procesados utilizando técnicas de \textit{data streaming}\cite{Fischer:2012:RVA:2245276.2245432}
	\item El sistema identificará potenciales comportamientos anómalos y generará las alarmas correspondientes.
	\item Con cada alarma, el sistema mostrará con que probabilidad se clasificó el evento (factor de riesgo).
	\item El sistema proveerá al usuario una interfaz web para la visualización de los eventos y las alarmas.
	\item El sistema almacenará información de eventos ocurridos en archivos de texto plano y en formato semi-estructurado.
\end{itemize}

\subsubsection*{No Funcionales}
\begin{itemize}
	\item Se proveerá un manual de instalación y configuración.
\end{itemize}
 
\subsection*{Exclusiones}
El proyecto no contempla la instalación del sistema en un ámbito de producción.

\subsection*{Supuestos}
Los datos necesarios para realizar pruebas pueden ser generados arbitrariamente. Además, se utilizarán datos provistos por el Centro de Cómputos de la Provincia de Santa Fe.

\subsection*{Criterios de Aceptación}
Se considera que el proyecto está aceptado cuando cumple en un 90\% los requisitos funcionales, con un nivel mínimo de aceptación del 75\%. Los prototipos deben tener implementadas todas las funcionalidades planificadas.
\section*{Metodología}

La metodología de desarrollo del proyecto será incremental e iterativa. Incremental porque varios componentes y funcionalidades del sistema se desarrollarán en momentos diferentes y serán integradas cuando sean completadas. Iterativa pues se invertirán esfuerzos en revisar constantemente partes del sistema, tanto para mejorar la calidad externa como interna del software\cite{ISOIEC9126}.

Dado que los requerimientos de los interesados pueden cambiar en el transcurso de la ejecución del proyecto, se propone utilizar un enfoque de desarrollo ágil. El mismo contempla la posibilidad de priorización y selección en los alcances y en las prioridades de las diferentes funcionalidades, y fundamentalmente promueve la entrega continua de software y la inclusión de los interesados en el proceso de desarrollo.

\section*{Plan de Tareas}

El proyecto se divide en 5 incrementos. A continuación se da una breve descripción de los mismos:

\paragraph{Incremento 1: Investigación preliminar}\
La primer parte del proyecto consiste en determinar que conjunto de tecnologías serán utilizadas, y elaborará una descripción a alto nivel de las diferentes componentes del sistema. Además se realizará una investigación sobre las técnicas de modelado de comportamiento existentes con el fin de determinar cuales serán implementadas.
\paragraph{Incremento 2: Modelado de comportamiento} \
En esta etapa se implementarán las técnicas de modelado de comportamiento seleccionadas en el incremento anterior. Se utilizarán solo los datos provenientes de la capa de transporte y se desarrollará un software con una interfaz web sencilla donde se muestren los principales parámetros del modelado.
\paragraph{Incremento 3: Detección de anomalías} \
Se implementarán las funcionalidades de detección de anomalías y se agregará al software mecanismos de generación de alarmas. Además se mejorará la interfaz de usuario.
\paragraph{Incremento 4: Modelado de comportamiento de subsistemas} \
Con el fin de mejorar la identificación de anomalías se incorporará al modelo de comportamiento información de los diferentes subsistemas que componen la infraestructura de red.
\paragraph{Incremento 5: Pruebas y documentación}
Se completará el desarrollo del software y se redactará la documentación necesaria. Una vez concluidas las pruebas, se elaborará un informe con los resultados obtenidos

\ \

Al finalizar la primera iteración de cada incremento se obtiene una herramienta de software con las funcionalidades descriptas y calidad de producto final, con excepción de la etapa de investigación preliminar donde se obtendrá un informe en soporte escrito o digital.



\begin{table}[htbp]
	\begin{center}	
		\begin{tabular}{|l|c|}
			\hline 
			Entregable & Fecha de entrega \\ \hline
			Informe de avance 1 & 04/10/2016 \\
			Informe de avance 2 & 08/11/2016 \\
			Informe de avance 3 & 16/12/2016 \\
			Informe de avance 4 & 14/02/2017 \\ \hline
		\end{tabular}
	\end{center}
	\caption{Fechas de entrega de informes de avance.}
	\label{table:informes}
\end{table}

\subsubsection*{Plan de Tareas}

La duración total del proyecto es de $466$ horas, con una dedicación de $20$ horas semanales. A continuación se detalla el plan de tareas.

\begin{enumerate}
	\setlength{\itemsep}{0pt}
	\setlength{\parskip}{0pt}
	\item \textbf{Investigación preliminar} (66hs)
	\begin{enumerate}
		\item Estudio comparativo de las tecnologías y métodos de modelado. (24hs)
		\item Diseño conceptual del sistema. (30hs)
		\item Documentación. (12hs)
	\end{enumerate}
	\item \textbf{Modelado de comportamiento} (84hs)
	\begin{enumerate}
		\item Instalación y configuración de la plataforma de desarrollo. (12hs)
		\item Implementación de funcionalidad de captura de tráfico de red. (24hs)
		\item Implementación de modelado de tráfico. (24hs)
		\item Implementación de interfaz web. (24hs)
	\end{enumerate}
	\item \textbf{Detección de anomalías} (98hs)
	\begin{enumerate}
		\item Implementación de funcionalidad de detección de anomalías. (30hs)
		\item Pruebas de implementación. (24hs)
		\item Implementación de módulo de alarmas. (24hs)
		\item Mejora de interfaz web. (20hs)
	\end{enumerate}
	\newpage
	\item \textbf{Modelado de comportamiento de subsistemas} (146hs)
	\begin{enumerate}
		\item Implementación de funcionalidad de captura de logs de subsistemas. (72hs)
		\begin{itemize}
			\item Servidores web (24hs).
			\item Gestores de base de datos (24hs).
			\item Firewalls (24hs).
		\end{itemize}
		\item Implementación de modelado de comportamiento de subsistemas. (30hs)
		\item Pruebas de implementación. (24hs)
		\item Implementación de funcionalidad de detección de anomalías y alarmas. (20hs)
	\end{enumerate}
	\item \textbf{Pruebas y documentación} (72hs)
	\begin{enumerate}
		\item Pruebas de detección. (20hs)
		\item Elaboración de informe de desempeño. (12hs)
		\item Elaboración de informe final. (40hs)
	\end{enumerate}
\end{enumerate}

\subsection*{Informes de avance}

Se presentarán 4 informes de avance en las fechas de finalización de cada etapa, detalladas en el Cuadro \ref{table:cronograma}. A continuación se detalla que información será incluida en cada informe:

\paragraph{Informe de avance 1} 
Contendrá los resultados obtenidos en los estudios comparativos de las tecnologías y las técnicas de modelado y detección y justificará la elección de las mismas. Además se incluirá una descripción general de la arquitectura del sistema.

\paragraph{Informe de avance 2}
Contendrá información sobre los criterios de selección de características para el modelado de tráfico de red. Se proveerá la guía de instalación y configuración de la plataforma. Además tendrá información sobre cambios realizados en los entregables anteriores y el desempeño del modelo implementado.

\paragraph{Informe de avance 3}
Contendrá información sobre los criterios de selección de características para el modelado del comportamiento de los subsistemas, y el desempeño del modelo implementado. Además se detallarán los cambios realizados en los entregables anteriores.

\paragraph{Informe de avance 4}
Se describirán las pruebas de integración del sistema. Además tendrá información sobre cambios realizados en los entregables anteriores y el desempeño general del sistema. También se incluirán los resultados de las pruebas de detección de la última iteración.

\begin{table}[htbp]
	\begin{center}	
		\begin{tabular}{|l|c|c|c|}
			\hline 
			Etapa & Inicio & Finalización & Duración \\ \hline
			Investigación preliminar & 01/09/2016 & 30/09/2016 & 4 semanas \\
			Modelado de comportamiento & 03/10/2016 & 04/11/2016 & 4 semanas \\
			Detección de anomalías & 07/11/2016 & 16/12/2016 & 5 semanas \\
			Modelado de comportamiento de subsistemas & 19/12/2016 & 18/02/2017 & 8 semanas \\
			Pruebas y documentación & 20/02/2017 & 25/03/2017 & 5 semanas \\ \hline
		\end{tabular}
	\end{center}
	\caption{Fechas estimativas de inicio y fin de actividades.}
	\label{table:cronograma}
\end{table}

\newpage

\subsection*{Diagrama de Gantt}

\begin{figure}[hb]
\begin{center}
\begin{ganttchart}[
	time slot format=isodate,
	x unit=.6mm,
	y unit title=.6cm,
	y unit chart=4mm,
	hgrid,
	link/.style={-latex}
	]{2016-09-01}{2017-03-29}
	\gantttitlecalendar{year, month} \\
	\ganttgroup{Tarea 1}{2016-09-01}{2016-09-30} \\
	\ganttbar[name=a11]{Actividad 1.1}{2016-09-01}{2016-09-08} \\
	\ganttbar[name=a12]{Actividad 1.2}{2016-09-10}{2016-09-30} \\
	\ganttbar[name=a13]{Actividad 1.3}{2016-09-10}{2016-09-30} \\
	\ganttlink{a11}{a12} 
	\ganttlink{a11}{a13} 
	\ganttgroup{Tarea 2}{2016-10-03}{2016-11-04} \\
	\ganttbar[name=a21]{Actividad 2.1}{2016-10-03}{2016-10-06} \\
	\ganttbar[name=a22]{Actividad 2.2}{2016-10-08}{2016-10-16} \\
	\ganttbar[name=a23]{Actividad 2.3}{2016-10-18}{2016-11-04} \\
	\ganttbar[name=a24]{Actividad 2.4}{2016-10-18}{2016-11-04} \\
	\ganttlink{a21}{a22}
	\ganttlink{a22}{a23}
	\ganttlink{a22}{a24} 
	\ganttgroup{Tarea 3}{2016-11-07}{2016-12-16} \\
	\ganttbar[name=a31]{Actividad 3.1}{2016-11-07}{2016-11-15} \\
	\ganttbar[name=a32]{Actividad 3.2}{2016-11-17}{2016-11-26} \\
	\ganttbar[name=a33]{Actividad 3.3}{2016-11-28}{2016-12-16} \\
	\ganttbar[name=a34]{Actividad 3.4}{2016-11-28}{2016-12-16} \\
	\ganttlink{a31}{a32}
	\ganttlink{a32}{a33}
	\ganttlink{a32}{a34}
	\ganttgroup{Tarea 4}{2016-12-19}{2017-02-18} \\
	\ganttbar[name=a41]{Actividad 4.1}{2016-12-19}{2017-01-17} \\
	\ganttbar[name=a42]{Actividad 4.2}{2017-01-19}{2017-02-10} \\
	\ganttbar[name=a43]{Actividad 4.3}{2017-01-19}{2017-02-10} \\
	\ganttbar[name=a44]{Actividad 4.4}{2017-02-12}{2017-02-18} \\
	\ganttlink{a41}{a42}
	\ganttlink{a41}{a43}
	\ganttlink{a42}{a44}
	\ganttlink{a43}{a44}
	\ganttgroup{Tarea 5}{2017-02-20}{2017-03-25} \\
	\ganttbar[name=a51]{Actividad 5.1}{2017-02-20}{2017-02-27} \\
	\ganttbar[name=a52]{Actividad 5.2}{2017-03-06}{2017-03-25} \\
	\ganttbar[name=a53]{Actividad 5.3}{2017-03-06}{2017-03-25}
	\ganttlink{a51}{a52}
	\ganttlink{a51}{a53}
\end{ganttchart}
\end{center}
\caption{Diagrama de Gantt del proyecto}
\end{figure}
%\end{landscape}

\section*{Riegos}

En esta sección se enumeran los riesgos identificados, indicadores y estrategia a adoptar según corresponda. Para realizar el análisis cualitativo de los riesgos se asigna una probabilidad de ocurrencia y un impacto a cada riesgo. Luego se priorizan según su \textit{severidad}. Los riesgos identificados son:

\begin{enumerate}
\item \textbf{No se pueden satisfacer las restricciones de performance:} la característica principal del sistema es la posibilidad de procesar tráfico en tiempo real. Puede darse un escenario donde las tecnologías disponibles no permitan alcanzar este requisito.

\textbf{Indicador:} se observan demoras o un progreso lento en las etapas de Núcleo y/o Integración. \\
\textbf{Probabilidad de ocurrencia:} Media. \\
\textbf{Impacto:} Muy alto. \\
\textbf{Estrategia a adoptar:} Mitigar (probabilidad).
Se revisan continuamente los entregables de la etapa de Diseño con el fin de mejorar su calidad.

\item \textbf{No se dispone del hardware necesario:} el servidor de aplicaciones necesario para el desarrollo de la aplicación no se encuentra disponible

\textbf{Indicador:} la etapa de Núcleo no puede comenzar. \\
\textbf{Probabilidad de ocurrencia:} Media. \\
\textbf{Impacto:} Alto. \\
\textbf{Estrategia a adoptar:} Mitigar (impacto).
Se utilizará la infraestructura provista por la facultad.

\item \textbf{Los módulos del sistema no pueden ser integrados correctamente:} aunque cada uno de los módulos cumpla con los requisitos funcionales del sistema, puede ocurrir que el rendimiento se vea degradado producto de los retardos que puede introducir las interfaces de comunicación entre los módulos.

\textbf{Indicador:} se observan demoras o un progreso lento en la etapa de Integración. \\
\textbf{Probabilidad de ocurrencia:} Baja. \\
\textbf{Impacto:} Muy alto. \\
\textbf{Estrategia a adoptar:} Mitigar (probabilidad).
Se revisan continuamente los entregables de la etapa de Integración con el fin de mejorar su calidad.

\item \textbf{Las tecnologías seleccionadas no pueden ser integradas:} se utilizarán un conjunto de tecnologías que deben coexistir para realizar el procesamiento de los datos. Es posible que se encuentren incompatibilidades entre alguna de ellas y no sea posible utilizarlas de forma conjunta.

\textbf{Indicador:} se observan demoras o un progreso lento en la etapa de Diseño. \\
\textbf{Probabilidad de ocurrencia:} Baja. \\
\textbf{Impacto:} Alto. \\
\textbf{Estrategia a adoptar:} Mitigar (impacto). \\
Se realiza un estudio comparativo de varias tecnologías para disponer de alternativas a las determinadas inicialmente.

\item \textbf{El sistema no puede ser probado en un entorno productivo:} la implementación del sistema en un entorno productivo implica un riesgo para los administradores de las redes. Por esto, es posible que no se disponga de un escenario real para las pruebas finales de integración.

\textbf{Probabilidad de ocurrencia:} Baja. \\
\textbf{Impacto:} Bajo. \\
\textbf{Estrategia a adoptar:} Aceptar (activamente). \\
Se generarán los datos necesarios para realizar pruebas de integración.

\end{enumerate}

\begin{center}

\end{center}

\subsection*{Análisis Riesgos}
En esta sección se muestra el análisis realizado de los riesgos del proyecto, luego se definen las estrategias a adoptar y los riesgos ordenados por importancia según su severidad.

\begin{center}
\vspace{15px}
\begin{tabular}{|c||c|c|c|c|c|}
	\hline
	& 1 & 2 & 3 & 4 & 5 \\ \hline
	\hline
	1 & \cellcolor[gray]{0.8} & \cellcolor[gray]{0.8} & \cellcolor[gray]{0.8} & \cellcolor[gray]{0.8} & \cellcolor[gray]{0.6} \\ \hline
	2 & \cellcolor[gray]{0.8} & \cellcolor[gray]{0.8} & \cellcolor[gray]{0.6} & \cellcolor[gray]{0.6} & \cellcolor[gray]{0.6} \\ \hline
	3 & \cellcolor[gray]{0.8} & \cellcolor[gray]{0.6} & \cellcolor[gray]{0.6} & \cellcolor[gray]{0.6} & \cellcolor[gray]{0.4} \\ \hline
	4 & \cellcolor[gray]{0.8} & \cellcolor[gray]{0.6} & \cellcolor[gray]{0.6} & \cellcolor[gray]{0.4} & \cellcolor[gray]{0.4} \\ \hline
	5 & \cellcolor[gray]{0.6} & \cellcolor[gray]{0.6} & \cellcolor[gray]{0.4} & \cellcolor[gray]{0.4} & \cellcolor[gray]{0.4} \\ \hline
\end{tabular}
\captionof{table}{Matriz probabilidad/impacto.}

\vspace{15px}

\begin{tabular}{|c|l|}
	\hline Severidad & Estrategia \\ \hline
	menor que 4 & Aceptar \\ \hline
	5 a 15 & Mitigar \\ \hline
	16 a 25 & Evitar \\ \hline
\end{tabular}
\captionof{table}{Estrategia a adoptar según severidad.}

\vspace{15px}

\begin{tabular}{|l||c|c|c|}
	\hline
	Riesgo & Imp. & \% ocur. & Sev. \\
	\hline
	No se pueden satisfacer las restricciones de performance.  & 5 & 3 & 15 \\
	No se dispone del hardware necesario. & 4 & 3 & 12 \\
	Los módulos del sistema no pueden ser integrados correctamente. & 5 & 2 & 10 \\
	Las tecnologías seleccionadas no pueden ser integradas. & 4 & 2 & 8 \\
	El sistema no puede ser probado en un entorno productivo. & 2 & 2 & 4 \\
	\hline
\end{tabular}
\captionof{table}{Lista de riesgos ordenados por severidad.}

\end{center}

\newpage

\section*{Presupuesto}

A continuación se detalla el presupuesto necesario para el desarrollo del proyecto. El valor de la infraestructura de hardware necesaria es aproximado y comprende una estación de trabajo para el desarrollo del sistema y un servidor dedicado con 3 interfaces de red. Las especificaciones técnicas se determinarán en la etapa de Diseño del proyecto.

\vspace{12px}

\begin{centering}

\begin{tabular}{l c c r}
\textbf{Infraestructura} & & & \\
Estación de trabajo & & & \$16000. \\
Servidor de desarrollo & & & \$10000. \\
\hline
\textbf{Servicios} & & &\\
Conexión a internet & & & \$4000 \\
\hline
\textbf{Recursos humanos} & \textbf{Costo por hora} & \textbf{Horas} \\
Diseñador & \$40 & 84 & \$3360 \\
Desarrollador & \$40 & 364 & \$14560 \\
Tester & \$40 & 40 & \$1600 \\
\hline
\textbf{Costo total} & & & \$\textbf{49520}.
\end{tabular}
\end{centering}




\newpage
 
\nocite{*}
\printbibliography
\end{document}




