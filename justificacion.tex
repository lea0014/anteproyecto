\section*{Justificación}
Las redes informáticas son esenciales para las operaciones diarias de cualquier empresa o institución pública. A medida que estas entidades crecen, también crece el volumen de datos que generan, demandando así una infraestructura de red de mayor porte. Debido a esto, mantener una buena performance en el funcionamiento general de un sistema que se encuentra constantemente en crecimiento se convierte en una tarea desafiante.

Algunas de las herramientas disponibles para monitoreo de redes brindan reportes basados en estadísticas. La empresa Cisco\textsuperscript{\textregistered} introduce una característica en algunos modelos de routers y switches llamada NetFlow, la cual permite recolectar información sobre el tráfico de red que atraviesa las distintas interfaces de los dispositivos. Sin embargo, estar limitado exclusivamente a routers y switches significa una desventaja pues en algunas ocasiones involucra una inversión en infraestructura considerable. Otros fabricantes introducen características similares pero con nombres diferentes: Traffic Flow de MikroTik, NetStream de HP, entre otros.
Otra alternativa para monitorear una interfaz de red es el analizador de paquetes \textit{tcpdump} (disponible en sistemas GNU/Linux), la cual permite observar una descripción en texto plano del contenido de los paquetes que pasan por una interfaz de red.

Una metodología diferente de monitoreo de red se relaciona con el BigData y consiste en almacenar todos los eventos que ocurren en la red y luego generar informes utilizando procesos \textit{batch} sobre el conjunto de datos. Esto no solo es un proceso lento sino que requiere almacenar grandes volúmenes de datos, y si bien brinda información que es de gran utilidad es sobre eventos que ocurrieron en el pasado.

Como parte fundamental de la infraestructura de cualquier tipo de empresa, la tarea de administración se ve facilitada si podemos contar con la información más actualizada posible sobre la evolución del tráfico de red. Es por esto que el monitoreo en tiempo real es un herramienta de gran utilidad pues permite contar con información importante, al instante. El proceso de toma de decisiones a la hora de invertir en infraestructura se ve facilitado con la información que puede extraerse observando la utilización de la infraestructura. Por ejemplo, una empresa podría decidir si invertir en una conexión a Internet de mayor velocidad o bloquear el acceso a algún servicio que el administrador detecta, causa gran demanda de ancho de banda. El monitoreo en tiempo real puede hacer también que el proceso de detección de problemas en una red sea rápida y sencilla.

Se propone entonces diseñar un sistema capaz de mostrar información acerca del tráfico que está atravesando una interfaz de red utilizando el sistema de procesamiento en tiempo real Apache Storm y las tecnologías asociadas al mismo. Analizando la salida de texto plano de la herramienta \textit{tcpdump} el sistema producirá en tiempo real informes acerca de los parámetros relevantes del tráfico de red, y mostrará informes gráficos según diferentes filtros como tráfico entrante y saliente, consumo de ancho de banda por protocolo, puerto o IP entre otros.
La empresa Twitter utiliza Storm para procesar grandes volúmenes de \textit{tweets} y determinar \textit{trends\footnote{Trend refiere a un tópico identificado por \textit{HashTags} que resulta sumamente popular en un momento dado.}}. El uso de esta tecnología hace que el sistema sea escalable y pueda adaptarse a redes que están en crecimiento constante. Otras compañías que hacen uso de Storm para proveer servicios son Yahoo!, Spotify, Yelp y Groupon por mencionar algunas importante.

Sin duda alguna, la cantidad de dispositivos conectados a Internet crece día a día y con ellos lo hace la demanda de la infraestructura de red necesaria para brindar un servicio de calidad. Contar con información del uso de las redes es fundamental para hacer estimaciones para inversiones a futuro, y permite determinar si los recursos disponibles están siendo utilizados de manera eficiente.

Finalmente, contar con esta información es una gran ayuda para mejorar la seguridad de las redes. Muchos de los ataques a redes informáticas son realizados por usuarios fuera de la red que no tienen acceso físico a la misma, pero muchos de los ataques más peligrosos son llevados a cabo por usuarios internos. Tener este tipo de información sobre el tráfico de la red puede ser una herramienta de enorme valor para identificar estos sucesos.

\section*{Objetivos}
\subsection*{Generales}
Diseñar un sistema que permita mostrar en tiempo real información acerca del tráfico de red que atraviesa una interfaz de red.
\subsection*{Específicos}
\begin{itemize}
\item Identificar un conjunto de tecnologías y herramientas adecuadas para el diseño del sistema.
\item Construir una herramienta de software que permita visualizar como varía el tráfico de red a través de tiempo.
\item Diseñar un sistema independiente de un dispositivo de hardware.
\item Describir conceptualmente la arquitectura del sistema.
\end{itemize}

\section*{Alcances}

\subsection*{Funcionales}
\begin{itemize}
	\item El sistema permitirá mostrar tiempo real el tráfico de red según criterios de filtrado dados.
	\item El sistema proveerá al usuario una interfaz para la visualización del tráfico de red.
	\item El sistema ofrecerá estadísticas generales de la red.
	\item Los informes mostrarán información de la Capa de Transporte y la Capa de Red\footnote{Modelo TCP/IP}.\cite{rfc791}\cite{rfc793}\cite{rfc1180}
\end{itemize}

\subsection*{No Funcionales}
\begin{itemize}
	\item El sistema será escalable. Las tecnologías seleccionadas permiten configurar un cluster de nodos de procesamiento en caso que el volumen de datos a procesar aumente.
	\item Se proveerá un manual de instalación y configuración.
\end{itemize}

\subsubsection{Exclusiones}
El proyecto no contempla la instalación del sistema en un ámbito de producción.
El sistema a desarrollar solo analizará el tráfico que atraviesa una única interfaz de red, y no contempla el caso de conexiones múltiples con balance de carga.
No se implementará ningún mecanismo de detección ataques o de patrones de comportamiento sospechoso.

\subsubsection{Supuestos}
Los datos necesarios para realizar pruebas pueden ser generados arbitrariamente.
El término \textit{tiempo real} refiere al denominado \textit{tiempo real blando}, es decir, no es necesario asegurar la ejecución de ciertas tareas o mostrar informes en el mismo instante en el que se generan los datos. La generación de los informes tendrá entonces demoras de cientos de milisegundos.

\subsubsection{Criterios de Aceptación}
Se considera que el proyecto está aceptado cuando cumple en un 90\% los requisitos funcionales, con un nivel mínimo de aceptación del 75\%. Los prototipos deben tener implementadas todas las funcionalidades planificadas.

