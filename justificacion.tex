\section*{Justificación}
Las redes informáticas son esenciales para que las empresas o instituciones públicas puedan realizar sus tareas diarias. A medida que estas entidades crecen también lo hace el volumen de datos que generan, demandando así una infraestructura de red con mayor tamaño. Debido a esto, mantener una buena performance en el funcionamiento general de un sistema que se encuentra constantemente en crecimiento se convierte en una tarea desafiante.

Algunas de las herramientas disponibles para monitoreo de redes brindan reportes basados en estadísticas. La empresa Cisco\textsuperscript{\textregistered} introduce una característica en algunos modelos de routers y switches llamada NetFlow, la cual permite recolectar información sobre el tráfico de red que atraviesa las distintas interfaces del dispositivo. Sin embargo, estar limitado exclusivamente a modelos particulares de routers y switches significa una desventaja pues en algunas ocasiones no se dispone del hardware mencionado, lo que implica una inversión en infraestructura considerable. Otros fabricantes introducen características similares a NetFlow, pero con nombres diferentes: Traffic Flow de MikroTik, NetStream de HP, entre otros.
Otra alternativa para monitorear el tráfico que pasa a través de una interfaz de red es el analizador de paquetes \textit{tcpdump} (disponible en sistemas GNU/Linux), el cual permite observar una descripción en texto plano del contenido de los paquetes que atraviesan la interfaz.

Disponer de la información más actualizada posible de los eventos que han ocurrido en la red facilita la tarea de administrar y diagnosticar fallas. Es por esto que el monitoreo en tiempo real es un herramienta de gran utilidad pues permite contar con información de una serie de métricas muy descriptivas al instante. El proceso de toma de decisiones a la hora de invertir en infraestructura puede realizarse de forma adecuada si se incluye la información que puede extraerse observando la utilización de la infraestructura actual. Por ejemplo, una empresa podría decidir si invertir en una conexión a Internet de mayor velocidad o limitar el ancho de banda disponible para ciertos servicios que hacen uso intensivo de las redes. El monitoreo y la visualización en tiempo real puede hacer también que el proceso de detección de problemas en una red sea rápido y sencillo.

Se propone entonces diseñar un sistema capaz de registrar los eventos ocurridos en capa de transporte y mostrar información acerca del tráfico de red que está siendo transmitido por una red. El mismo desarrollará utilizando técnicas llamadas \textit{Real-Time Stream Processing} para analizar la salida de texto plano del analizador de paquetes \textit{tcpdump} y producir en tiempo real informes acerca de los parámetros de performance más importantes\footnote{IP Performance Metrics Working Group https://datatracker.ietf.org/wg/ippm/documents/ (IPPM)}
. El uso de estas técnicas permite diseñar un sistema escalable el cual pueda adaptarse a redes que están en crecimiento constante.

Sin duda alguna, la cantidad de dispositivos conectados a Internet crece día a día y con ellos lo hace la demanda de la infraestructura de red necesaria para brindar un servicio de calidad. Contar con información del uso de las redes es fundamental para hacer estimaciones para inversiones a futuro, y permite determinar si los recursos disponibles están siendo utilizados de manera eficiente.

Finalmente, contar con esta información es una gran ayuda para mejorar la seguridad de las redes. Muchos de los ataques a redes informáticas son realizados por usuarios fuera de la red que no tienen acceso físico a la misma, pero muchos de los ataques más peligrosos son llevados a cabo por usuarios internos. Tener este tipo de información sobre el tráfico de la red puede ser una herramienta de enorme valor para identificar estos sucesos.

\section*{Objetivos}
\subsection*{Generales}
Diseñar un sistema que permita monitorear en tiempo real parámetros de performance del tráfico que atraviesa una interfaz de red.
\subsection*{Específicos}
\begin{itemize}

\item Identificar un conjunto de tecnologías y herramientas adecuadas para el diseño del sistema.
\item Diseñar un sistema de monitoreo de redes que funcione en \textit{commodity hardware}.
\item Construir una herramienta de software que permita visualizar cómo varían los parámetros de performance de la red (ver Alcances).
\item Elaborar un documento que describa la arquitectura del sistema.
\end{itemize}

\section*{Alcances}

\subsection*{Funcionales}
\begin{itemize}
	\item El sistema permitirá visualizar en tiempo real los parámetros de performance de la red. Se podrá monitorear el tráfico a nivel de \textit{host} (capa de red) y a nivel de servicios (capa de transporte)
	\item Los informes mostrarán información de la Capa de Transporte y la Capa de Red\footnote{Modelo TCP/IP}. \cite{rfc791}\cite{rfc793}\cite{rfc1180}
	\item El sistema proveerá al usuario una interfaz para la visualización de los informes. Estos contaran con los siguientes parámetros:
	\begin{itemize}
		\item Utilización de la red
		\item Paquetes por segundo transmitidos
		\item Cantidad de peticiones de conexión
		\item Cantidad de conexiones realizadas
		\item Cantidad de conexiones rechazadas y expiradas
		\item Cantidad de conexiones activas
		\item Tráfico entrante y saliente por host
		\item Tráfico entrante y saliente por proceso
	\end{itemize}
	\item El sistema ofrecerá estadísticas generales de uso de la red.
	\item El sistema almacenará información de las conexiones establecidas.
	\item El sistema no identificará patrones o generará alarmas.
\end{itemize}

\subsection*{No Funcionales}
\begin{itemize}
	\item El sistema será escalable. Las tecnologías seleccionadas permiten configurar un cluster de nodos de procesamiento en caso que el volumen de datos a procesar aumente.
	\item Se proveerá un manual de instalación y configuración.
\end{itemize}
 
\subsubsection{Exclusiones}
El proyecto no contempla la instalación del sistema en un ámbito de producción. El sistema a desarrollar solo analizará el tráfico que atraviesa una única interfaz de red, y no contempla el caso de conexiones múltiples con balance de carga, ni la utilización de múltiples nodos configurados en modo cluster. No se implementará ningún mecanismo de detección ataques o de patrones de comportamiento sospechoso.

\subsubsection{Supuestos}
Los datos necesarios para realizar pruebas pueden ser generados arbitrariamente.
El término \textit{tiempo real} refiere al denominado \textit{tiempo real blando}, es decir, no es necesario asegurar la ejecución de ciertas tareas o mostrar informes en el mismo instante en el que se generan los datos. La generación de los informes tendrá entonces demoras de cientos de milisegundos.

\subsubsection{Criterios de Aceptación}
Se considera que el proyecto está aceptado cuando cumple en un 90\% los requisitos funcionales, con un nivel mínimo de aceptación del 75\%. Los prototipos deben tener implementadas todas las funcionalidades planificadas.